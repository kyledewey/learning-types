\section{Introduction}
This document covers the process of \emph{typechecking}, that is, of determining if an input program is well-typed or ill-typed.
This assumes the reader is familiar with context-free grammars and abstract syntax trees.
This is intended to be a quick introduction for someone who needs to implement a compiler fast, as opposed to a full introduction to type theory.

\subsection{Expressions and Statements}
In most programming languages, programs are composed primarily of \emph{expressions} and \emph{statements}.
Expressions produce a value, whereas statements do not produce values.
Usually, statements are useful for some sort of effect they have on the code.
For example, $1$, $1 + 2$, and $x * 5$ are all expressions; these evaluate down to some value.
In contrast, \texttt{int x = 5;} is a statement; this has the effect of declaring a variable and initializing it with the value of $5$, and does not produce a value overall.

If this still isn't clear, ask yourself: ``can I assign this into a variable?''
If yes, it's an expression, if not, it's a statement (for our purposes, anyway).
For example, dovetailing off of the previous examples, \texttt{myVar = 1}, \texttt{myVar = 1 + 2}, and \texttt{myVar = x * 5} all make sense, whereas \texttt{myVar = int x = 5;} does not.

Note that statements can contain expressions within them.
Using the prior examples, $5$ is an expression, but \texttt{int x = 5;} is a statement.
Some languages also permit statements to be embedded in expressions.
Exactly what is permissible is defined by the language's grammar; most languages will have production rules for expressions and statements.

\subsection{Types}
Most languages have the concept of \emph{types}.
Types define the kinds of values which are possible in the language, along with the operations which are permissable on specific values.
For example, consider the following Java code:

\begin{verbatim}
int x = 1;
int y = 2;
int z = x + y;
\end{verbatim}

Java knows that \texttt{x}, \texttt{y}, and \texttt{z} are all variables of type \texttt{int}, and that \texttt{+} is a valid operation that can be performed with two values of type \texttt{int} (specifically \texttt{x} and \texttt{y} in the example above).
Moreover, the result of the \texttt{+} operation is \texttt{int}.

\subsubsection{Statically-Typed Languages}
Java is an example of a \emph{statically-typed language}.
This means that the types of all variables are known at compile time.
In statically-typed languages, programs with type errors (i.e., types are used in an inconsistent manner) are rejected (i.e., they fail to compile).

\subsubsection{Dynamically-Typed Languages}
Python (older variants) and JavaScript are examples of \emph{dynamically-typed languages}.
This means that types are associated with \emph{values}, not variables.
This is far less restrictive than with statically-typed languages.
For example, the following is legal Python code:

\begin{verbatim}
x = 7
x = "foo"
\end{verbatim}

With this program, while \texttt{7} is of type \texttt{int} and \texttt{``foo''} is of type \texttt{String}, \texttt{x} itself has no fixed type associated with it; \texttt{x}'s type depends on whichever value it is holding at the moment.
This program cannot be written in any statically-typed language, because \texttt{x}'s type isn't permitted to change in a statically-typed language.

Dynamically-typed languages are less restrictive than statically-typed languages, but this comes with a cost.
For one, performance-wise, generally the more information the compiler has at compile time, the better it can optimize your code.
For instance, if the compiler knows that \texttt{x} is an \texttt{int}, and \texttt{int} values are 64 bits on your platform, then it can allocate exactly 32 bits for your variable.
However, if the compiler doesn't know the type of \texttt{x}, it will usually have to allocate space to hold a `pointer to a `box'', where the box itself holds the value.
This means we'd need 64 bits for the pointer (assuming a 64 bit platform), which itself would point to a box.
The box would need to be at least 64 bits large for the integer, and is usually a bit larger.
Moreover, we add an extra level of indirection for every access to \texttt{x}; to get the integer out, we no longer just look up \texttt{x}, we also have to dereference \texttt{x} (two memory lookups as opposed to one).

Beyond performance, dynamically-typed languages can be overly permissive, to the point where they make life difficult.
If a type error is possible in a dynamically-typed program, then we must find just the right input that will trigger it.
A bug is present, but it's hidden from us.


\subsection{Static Typechecking}
For our purposes, we will assume we're working with a statically-typed language, and are typechecking the program at compile time.
The process of typechecking is usually phrased as a series of rules which explain how to derive what the type of a given program is.
For example, with $1 + 3$, we would need (at least) two rules working in conjunction:
\begin{itemize}
\item The type of any integer (e.g., $1$ and $3$) is \texttt{int}
\item The type of $e_1 + e_2$, where $e_1$ is \texttt{int} and $e_2$ is \texttt{int}, is \texttt{int}
\end{itemize}

Rules can be specified in natural languages (as English is mostly used above), but this tends to be problematic.
Natural languages tend to be both verbose and imprecise, leading to lengthy, unclear descriptions.
Instead, it is usually preferable to use \emph{inference rules}, which can be seen as a way of writing an unambiguous specification using math.
The next section will introduce inference rules via example.

