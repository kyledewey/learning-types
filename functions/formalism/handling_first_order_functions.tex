\section{Handling First-Order Functions}
We'll first cover how to handle first-order functions, and then add in higher-order functions.

\subsection{Syntax}
We will first introduce first-order functions.
As part of this, we need to modify our syntax, shown below.
For convenience, we'll introduce the notation $\vec{a}$, which means we have zero or more $a$'s; this has the same meaning as $a^*$ in standard EBNF, but it avoids some ambiguities.

\begin{gather*}
  x \in \mtt{Variable} \qquad i \in \mtt{Integer} \qquad fn \in \mtt{FunctionName}
\end{gather*}
\begin{align*}
  \tau \in \mtt{Type} &::= \kw{int} \alt \kw{bool}\\
  e \in \mtt{Exp} &::= x \alt i \alt \kw{true} \alt \kw{false} \alt e_1 \;\&\&\; e_2 \alt e_1 + e_2 \alt e_1 < e_2 \alt fn(\vec{e})\\
  s \in \mtt{Stmt} &::= \kw{let } x: \tau = e \alt x = e\\
  f \in \mtt{Function} &::= \tau_1\;fn(\overrightarrow{\tau_2\;x}) \{ \vec{s}\;e \}\\
  p \in \mtt{Program} &::= \vec{f}
\end{align*}

Compared to some prior changes, this one is fairly dramatic.
Notably:
\begin{itemize}
\item We now have function names ($fn$) as a primitive thing.
\item We now have first-order functions ($f$).
  First order functions have a return type ($\tau_1$).
  These take zero or more parameters, each with their own type ($\overrightarrow{\tau_2\;x}$).
  The body of the function consistes of zero or more statements ($\vec{s}$), followed by a single expression ($e$).
  The intention here is that a function returns whatever $e$ evaluates to.
  By forcing functions to end with an expression, we syntactically ensure that functions always have a return.
  If we instead made a \texttt{return} statement, we'd need to ensure that the function contains a \texttt{return} statement; this is doable, and real languages do this, but it's extra relatively uninteresting work.
\item We now have function calls, which take zero or more parameters $fn(\vec{e})$.
\item Programs are now defined as a series of zero or more first-order functions.
  The intention is that one of these functions (traditionally named \texttt{main}) serves as the entry point of the program.
\end{itemize}

\subsection{Type Rules}
As before, we will need a type environment mapping variables to types:
\begin{align*}
  \Gamma \in \mtt{TypeEnv} &= \mtt{Variable} \to \mtt{Type}
\end{align*}

However, this time around, the type environment alone won't be enough.
Consider function calls.
When calling a function, we need to know a few things:
\begin{itemize}
\item Whether or not the function is defined (its a type error to call a nonexistant function)
\item The number of parameters the function takes (if we pass too many or too few, this should be a type error)
\item The types of the parameters the function takes (if we pass arguments of the wrong type, this should be a type error)
\item The return type of the function (if we pass the right number of parameters of the right types, then we need to know what we get back)
\end{itemize}

It's most convenient to gather all this information into a single data structure before we execute any typechecking rules.
This data structure is commonly referred to as a \emph{symbol table}, which is a fancy term for a data structure that holds program information which can be derived directly from the abstract syntax tree.
For our purposes, the following definition suffices:

\begin{align*}
  st \in \mtt{SymbolTable} &= \mtt{FunctionName} \to \mtt{Type} \times \overrightarrow{\mtt{Type}}
\end{align*}

The above definition says that a symbol table maps function names to tuples (specifically 2-tuples: pairs).
The first element of the pair is a type, which encodes the return type of the given function.
The second element of the pair is a list of types, which encode the parameter types to the function.
The symbol table can be derived using the following Python-like pseudocode:
\begin{verbatim}
symbol_table = dict()
for f in functions:
  if f.name in symbol_table:
    raise TypeError("duplicate function name")
  param_names = set([entry.name for entry in f.parameters])
  if len(param_names) != len(f.parameters):
    raise TypeError("duplicate parameter name")
  dict[f.name] = (f.return_type, [entry.type for entry in f.parameters])
\end{verbatim}

\noindent
The above code disallows two functions to have the same name, and similarly disallows two formal parameters of the same function to have the same name.
Other than this, it basically just summarizes all the function parameter and return types.

With the symbol table in mind, we can define our typing rules.
Throughout the duration of typechecking, the symbol table ($st$) never changes.
As such, we will treat $st$ as if it's a global variable.
Pedantically, $st$ should be threaded through all the rules, but because it never changes, this just adds a lot of noise to the rules without giving any real benefit.
Without further ado, we define the rules below.

\subsubsection{Rules for Expressions}
The original rules for expressions are unchanged, though \textsc{less-than} has been renamed \textsc{lt} for space.
We now have a rule handling first-order function calls (\textsc{fo-call}), which makes use of $st$.
\begin{center}
  \begin{tabular}{ccc}
    \infer[(\textsc{integer})]
      {\typeof{i}{\Gamma}{\kw{int}}}
      {}
    &
    \infer[(\textsc{true})]
      {\typeof{\kw{true}}{\Gamma}{\kw{bool}}}
      {}
    &
    \infer[(\textsc{false})]
      {\typeof{\kw{false}}{\Gamma}{\kw{bool}}}
      {}
      \\
      \\
    \infer[(\textsc{and})]
      {\typeof{e_1 \;\&\&\; e_2}{\Gamma}{\kw{bool}}}
      {\typeof{e_1}{\Gamma}{\kw{bool}} \quad \typeof{e_2}{\Gamma}{\kw{bool}}}
    &
    \infer[(\textsc{plus})]
      {\typeof{e_1 + e_2}{\Gamma}{\kw{int}}}
      {\typeof{e_1}{\Gamma}{\kw{int}} \quad \typeof{e_2}{\Gamma}{\kw{int}}}
    &
    \infer[(\textsc{lt})]
      {\typeof{e_1 < e_2}{\Gamma}{\kw{bool}}}
      {\typeof{e_1}{\Gamma}{\kw{int}} \quad \typeof{e_2}{\Gamma}{\kw{int}}}
    \\
    \\
    \infer[(\textsc{var})]
      {\typeof{x}{\Gamma}{\tau}}
      {x \in \texttt{dom}(\Gamma) \quad \tau = \Gamma[x]}
    &
    \infer[(\textsc{fo-call})]
      {\typeof{fn(\vec{e})}{\Gamma}{\tau_1}}
      {fn \in \texttt{dom}(st) \quad
        (\tau_1, \vec{\tau_2}) = st[fn] \vspace{0.05in}\\
        {|\vec{e}| = |\vec{\tau_2}|} \quad
        \overrightarrow{\typeof{e}{\Gamma}{\tau_2}}}
    &
  \end{tabular}
\end{center}

\textsc{fo-call} above a lot of work, specifically:
\begin{itemize}
\item It checks that the called function ($fn$) was defined ($fn \in \texttt{dom}(st)$).
\item It looks up the expected return type and parameter types of the function ($(\tau_1, \vec{\tau_2}) = st[fn]$).
\item It checks that the number of actual parameters matches up with the number of expected parameters ($|\vec{e}| = |\vec{\tau_2}|$).
  Note that $|a|$ denotes the cardinality of $a$ here.
\item It checks that the types of the actual parameters ($\vec{e}$) match up with the types of the expected parameters ($\vec{\tau_2}$).
\end{itemize}
If any of the above checks fail, the typechecking of the call will fail.

\subsubsection{Rules for Statements}
The rules for statements are unchanged, and have been duplicated below for convenience.
$\vdash_s$ denotes typing rules for statements.
\begin{center}
  \begin{tabular}{cc}
    \infer[(\textsc{let})]
      {\typeofs{\kw{let } x: \tau = e}{\Gamma}{\Gamma[x \mapsto \tau]}}
      {\typeof{e}{\Gamma}{\tau}}
    &
    \infer[(\textsc{assign})]
      {\typeofs{x = e}{\Gamma}{\Gamma}}
      {x \in \texttt{dom}(\Gamma) \quad
        \tau = \Gamma[x] \quad
        \typeof{e}{\Gamma}{\tau}}
  \end{tabular}
\end{center}

\subsubsection{Rules for Functions}
We now need a typing rule that handles functions.
Since our functions are all defined at the toplevel, and because we have no global variables in this language, our typing rules for functions do not need to take a type environment $\Gamma$; $\Gamma$ will always be empty initially, as no variables can possibly be in scope.
For similar reasons, it doesn't make sense to return a type environment or a type; functions live in the symbol table ($st$), which exists independently of thiese rules.
We define the typing rules below:

\begin{center}
  \begin{tabular}{c}
    \infer[(\textsc{function})]
      {\tau_1\;fn(\overrightarrow{\tau_2\;x}) \{ \vec{s}\;e \}}
      {\Gamma_1 = \overrightarrow{[x \mapsto \tau_2]} \quad
        \overrightarrow{\Gamma_1 \vdash_s s} : \Gamma_2 \quad
        \typeof{e}{\Gamma_2}{\tau_1}}
  \end{tabular}
\end{center}

\noindent
The above rule says:
\begin{itemize}
\item Construct a type environment ($\Gamma_1$) from the parameters to the function.
  Each parameter name ($x$) becomes a key, and the parameter's corresponding type ($\tau_2$) becomes a value.
\item Using $\Gamma_1$ as an initial type environment, typecheck the statements ($\vec{s}$), yielding a new type environment ($\Gamma_2$).
\item Typecheck the expression ($e$) underneath $\Gamma_2$.
  The type of $e$ should match up with the return type the user specified ($\tau_1$)
\end{itemize}

If we apply \textsc{function} to each function in the program, then we will end up typechecking the whole program.
If any application of \textsc{function} fails, then there is a type error in the program.

