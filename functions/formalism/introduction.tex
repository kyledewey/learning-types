\section{Introduction}
All modern programming languages have some concept of a bit of code that can be called on demand.
These bits go by a lot of names, each with their own subtly different meanings, including functions, methods, subroutines, and procedures.
In this handout, we'll go over how to typecheck functions.

Specifically, we will cover two kinds of functions, which are not mutually-exclusive:
\begin{itemize}
\item First-order functions
\item Higher-order functions
\end{itemize}

\noindent
First-order functions generally exist at the toplevel (or near the toplevel) of a program, and are required to have a name.
All functions in C are first-order functions.
In contrast, higher-order functions generally do not have an associated name, and these can be created anywhere.
Higher-order functions allow us to treat functions as data: whole functions can be passed as parameters, returned from other functions, or saved in a data structure.
We assume the reader already has some familiarity with higher-order functions.

The rest of this handout covers how to handle these two kinds of functions.
