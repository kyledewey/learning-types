\section{Handling Higher-Order Functions}
In this section, we add support for higher-order functions.

\subsection{Syntax}
We first modify our syntax for higher-order functions:

\begin{gather*}
  x \in \mtt{Variable} \qquad i \in \mtt{Integer} \qquad fn \in \mtt{FunctionName}
\end{gather*}
\begin{align*}
  \tau \in \mtt{Type} &::= \kw{int} \alt \kw{bool} \alt \vec{\tau_1} \Rightarrow \tau_2\\
  e \in \mtt{Exp} &::= x \alt i \alt \kw{true} \alt \kw{false} \alt e_1 \;\&\&\; e_2 \alt e_1 + e_2 \alt e_1 < e_2 \alt fn(\vec{e})\\
  &\lalt (\overrightarrow{\tau\;x}) \Rightarrow e \alt e_1(\vec{e_2})\\
  s \in \mtt{Stmt} &::= \kw{let } x: \tau = e \alt x = e\\
  f \in \mtt{Function} &::= \tau_1\;fn(\overrightarrow{\tau_2\;x}) \{ \vec{s}\;e \}\\
  p \in \mtt{Program} &::= \vec{f}
\end{align*}

\noindent
There are three notable changes above:
\begin{itemize}
\item We've added a new type ($\vec{\tau_1} \Rightarrow \tau_2$).
  Since higher-order functions are themselves values, they need their own type.
  This type is \emph{recursively defined}; it contains other types inside of it (the $\vec{\tau_1}$ and $\tau_2$).
  Specifically, this type encodes the expected parameter types of the higher-order function ($\vec{\tau_1}$), as well as the return type ($\tau_2$).
\item We've added a new expression: ($(\overrightarrow{\tau\;x}) \Rightarrow e$).
  This creates a new higher-order function with the parameters and their corresponding types specified by $(\overrightarrow{\tau\;x})$.
  The function returns whatever $e$ returns.
\item We've added another new expression: $e_1(\vec{e_2})$.
  This calls a higher-order function.
  The intention is that $e_1$ evaluates down to a higher-order function, and $\vec{e_2}$ specifies the parameters to the higher-order function.
\end{itemize}

\subsection{Type Rules}
Now that the syntax has been tweaked, we can focus on the corresponding typing rules.
Surprisingly, very little has to change to handle higher-order functions.
In fact, we need only add the rules specific to our new expressions; everything else stays the same.
These rules are below, where \textsc{ho} stands for ``higher-order''.

\begin{center}
  \begin{tabular}{cc}
    \infer[(\textsc{ho-def})]
      {\typeof{(\overrightarrow{\tau_1\;x}) \Rightarrow e}{\Gamma_1}{\vec{\tau_1} \Rightarrow \tau_2}}
      {\texttt{allDistinct}(\vec{x}) \quad
        \Gamma_2 = \Gamma_1\overrightarrow{[x \mapsto \tau_1]} \quad
        \typeof{e}{\Gamma_2}{\tau_2}}
    &
    \infer[(\textsc{ho-call})]
      {\typeof{e_1(\vec{e_2})}{\Gamma}{\tau_2}}
      {\typeof{e_1}{\Gamma}{\vec{\tau_1} \Rightarrow \tau_2} \quad
        |\vec{\tau_1}| = |\vec{e_2}| \quad
        \overrightarrow{\typeof{e_2}{\Gamma}{\tau_1}}}
  \end{tabular}
\end{center}

\noindent
A description of the above rules follows:
\begin{itemize}
\item \textsc{ho-def} first checks to make sure that none of the variables are duplicated in the parameters using $\texttt{allDistinct}(\vec{x})$.
  Here, \texttt{allDistinct} is a \emph{helper function} (yes, we can have these in math, too).
  In this case, the meaning of \texttt{allDistinct} should be clear, though pedantically \texttt{allDistinct}'s definition should be provided.
  We create a new type environment $\Gamma_2$ from $\Gamma_1$, adding in all the variable/type bindings ($\overrightarrow{[x \mapsto \tau_1]}$) introduced in the function (via $\Gamma_2 = \Gamma_1\overrightarrow{[x \mapsto \tau_1]}$).
  We then typecheck $e$ underneath $\Gamma_2$, yielding a return type for our function ($\tau_2$).
  We finally bundle everything together into a single function type ($\vec{\tau_1} \Rightarrow \tau_2$).
\item \textsc{ho-call} first checks that expression $e_1$ is a function type ($\typeof{e_1}{\Gamma}{\vec{\tau_1} \Rightarrow \tau_2}$).
  From there, it checks that the function was given the correct number of parameters ($|\vec{\tau_1}| = |\vec{e_2}|$).
  If so, it will check that the types of the actual parameters ($\vec{e_2}$) match the expected parameter types ($\vec{\tau_1}$), via $\overrightarrow{\typeof{e_2}{\Gamma}{\tau_1}}$.
  Finally, it yields the result type $\tau_2$, corresponding to the return type of the function.
\end{itemize}

\noindent
With that, we've added both first-order and higher-order functions to this language.
